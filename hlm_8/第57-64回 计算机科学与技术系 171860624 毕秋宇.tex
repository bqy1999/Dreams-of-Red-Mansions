%what is the homework? put it here.

%%%%%%%%%%%%%%%%%%%%%%%%%%%%%%%%%%%%%%%%%%%%%%%%%%%%%%%%%%%%%%%%
%                         My Template:                         %
%%%%%%%%%%%%%%%%%%%%%%%%%%%%%%%%%%%%%%%%%%%%%%%%%%%%%%%%%%%%%%%%

%Code(C++): \begin{lstlisting}
%Algorithm:
%\begin{breakablealgorithm}
%  \caption{?statement}
%  \begin{algorithmic}[?number]
%    \Require ?input
%    \Ensure ?output
%    \Procedure{Equal}{?parameters}
%      \State ?blabla
%    \EndProcedure
%  \end{algorithmic}
%\end{breakablealgorithm}

%Itemlisting: \begin{itemize} or \begin{enumerate}[label=(\alph*)]

%Math equation: \begin{align*}

%Table: \begin{tabular}{|c|c|c|}
%           blabla | blabla | blabla \\
%           ......
%Picture: \centerline{\includegraphics[scale=X]{FileName}

%%%%%%%%%%%%%%%%%%%%%%%%%%%%%%%%%%%%%%%%%%%%%%%%%%%%%%%%%%%%%%%%
%                         Title START!                         %
%%%%%%%%%%%%%%%%%%%%%%%%%%%%%%%%%%%%%%%%%%%%%%%%%%%%%%%%%%%%%%%%
\documentclass[11pt, a4paper, UTF8]{ctexart}

\input{preamble}

\title{经典悦读《红楼梦》}
\me{毕秋宇}{计算机科学与技术系}{171860624}
\date{\today}

\begin{document}
\maketitle
% \noplagiarism

%%%%%%%%%%%%%%%%%%%%%%%%%%%%%%%%%%%%%%%%%%%%%%%%%%%%%%%%%%%%%%%%
%                       Homework START!                        %
%%%%%%%%%%%%%%%%%%%%%%%%%%%%%%%%%%%%%%%%%%%%%%%%%%%%%%%%%%%%%%%%
\beginthishw
%%%%%%%%%%%%%%%%%%%%
\begin{problem}[第57-64回]
\end{problem}

%\begin{remark}
%
%\end{remark}
\begin{solution}
    \

    \large 总有人说:"自古女子多痴情",其实也不尽如此。探春对儿女之事不甚上心,大观园改革却颇有见解;宝玉虽满腹才华,却一心扑在儿女情长之上。五十七回中“惠紫鹃情辞试莽玉”,更是把这一反差显现得淋漓精致。宝玉一听说黛玉要离他去往苏州,竟摄魂落魄一副半死不活的样子,把贾府上下弄的人心惶惶,真可见宝玉之“莽”。把儿女情长当成赖以生存的命根子,可谓莽子真重情。

    第六十及六十一回中又写了大观园中的一桩冤案,柳五儿被诬陷偷取茯苓霜,百口莫辩。平儿主审却发现原是彩云所窃,可谓是投鼠忌器、有所顾忌,一方面茯苓霜是个珍奇玩意儿非常难得,又怕伤害探春毁了姐弟间的和气,最终只好大事化小、小事化了糊弄了过去。从这也可以看出探春的改革注定是无效的,一起冤案因为自身缘故无法追责,她提的那些个改革方案又有多少能够落到实处,解决两府痛点,真令人心生疑惑。
\end{solution}

% \begin{problem}[TJ: 9.23]
% \end{problem}

%\begin{remark}
%
%\end{remark}

% \begin{solution}
% \end{solution}

% \begin{problem}[TJ: 10.11]
% \end{problem}

%\begin{remark}
%
%\end{remark}

% \begin{solution}
% \end{solution}

% \begin{problem}[TJ: 11.5]
% \end{problem}

%\begin{remark}
%
%\end{remark}

% \begin{solution}
% \end{solution}

% \begin{problem}[optional question 1: TJ: 11.17]
% \end{problem}

%\begin{remark}
%
%\end{remark}

% \begin{solution}
% \end{solution}

% \begin{problem}[optional question 2]
% \end{problem}

%\begin{remark}
%
%\end{remark}

% \begin{solution}
% \end{solution}
%%%%%%%%%%%%%%%%%%%%
%\newpage
%%%%%%%%%%%%%%%%%%%%


%%%%%%%%%%%%%%%%%%%%%%%%%%%%%%%%%%%%%%%%%%%%%%%%%%%%%%%%%%%%%%%%
%                      Correction START!                       %
%%%%%%%%%%%%%%%%%%%%%%%%%%%%%%%%%%%%%%%%%%%%%%%%%%%%%%%%%%%%%%%%
% \begincorrection
%%%%%%%%%%%%%%%%%%%%
%\begin{problem}[]

%\end{problem}

%\begin{cause}
%
%\end{cause}

%\begin{revision}

%\end{revision}
%%%%%%%%%%%%%%%%%%%%
%\newpage
%%%%%%%%%%%%%%%%%%%%


%%%%%%%%%%%%%%%%%%%%%%%%%%%%%%%%%%%%%%%%%%%%%%%%%%%%%%%%%%%%%%%%
%                       Feedback START!                        %
%%%%%%%%%%%%%%%%%%%%%%%%%%%%%%%%%%%%%%%%%%%%%%%%%%%%%%%%%%%%%%%%
% \beginfb
%\begin{itemize}
%
%\end{itemize}


%%%%%%%%%%%%%%%%%%%%%%%%%%%%%%%%%%%%%%%%%%%%%%%%%%%%%%%%%%%%%%%%
%                        Homework END!                         %
%%%%%%%%%%%%%%%%%%%%%%%%%%%%%%%%%%%%%%%%%%%%%%%%%%%%%%%%%%%%%%%%
\end{document}

