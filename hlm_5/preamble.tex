%%%%%%%%%%%%%%%%%%%%%%%%%%%%%%%%%%%
% File: preamble.tex
%%%%%%%%%%%%%%%%%%%%%%%%%%%%%%%%%%%
\usepackage{geometry}
\geometry{left = 1cm, right = 1cm, top = 1.5cm, bottom = 1.5cm}

\ProvidesPackage{zhfontcfg}
\usepackage{fontspec,xunicode}
\usepackage{xeCJK}
\usepackage{CJKutf8}
\defaultfontfeatures{Mapping=tex-text}
\XeTeXlinebreaklocale "zh"
\XeTeXlinebreakskip = 0pt plus 1pt minus 0.1pt
% Set fonts commands
% \newcommand\fontnamehei{文泉驿正黑}
% \newcommand\fontnamesong{文鼎PL新宋}
% \newcommand\fontnamekai{AR PL UKai CN}
% \newcommand\fontnamemono{Bitstream Vera Sans Mono}
% \newcommand\fontnameroman{Bitstream Vera Serif}
% \newcommand{\song}{\fontnamesong}
% \newcommand{\hei}{\fontnamehei}
% \newfontinstance\KAI {\fontnamekai}
% \newcommand{\kai}{\fontnamekai}
% \newCJKfontfamily\kai{FandolKai-Regular.otf}
% \newCJKfontfamily\hei{FandolHei-Regular.otf}
% \newCJKfontfamily\song{FandolSong-Regular.otf}
% \newCJKfontfamily\fang{FandolFang-Regular.otf}
% \setCJKfamilyfont{song}[BoldFont=FandolSong-Bold.otf]{FandolSong-Regular.otf}
% \setCJKfamilyfont{hei}{FandolHei-Regular.otf}
% \setCJKfamilyfont{kai}{FandolKai-Regular.otf}
\newcommand{\song}{\CJKfamily{song}}
\newcommand{\hei}{\CJKfamily{hei}}
\newcommand{\kai}{\CJKfamily{kai}}
\newcommand{\fs}{\CJKfamily{fs}}
\newcommand{\tqs}{\textquotesingle}

\defaultfontfeatures{
    Path = /usr/local/texlive/2018/texmf-dist/fonts/opentype/public/fontawesome/ }
\usepackage{fontawesome}
\newcommand{\me}[3]{\author{{\bfseries 姓名:}\underline{#1}\hspace{2em}{\bfseries 院系:}\underline{#2}\hspace{2em}{\bfseries 学号:}\underline{#3}}}
% \newcommand{\me}[2]{\author{{\bfseries 姓名:}\underline{#1}\hspace{2em}{\bfseries 学号:}\underline{#2}}}

% Always keep this.
\newcommand{\noplagiarism}{
    \begin{center}
        \fbox{\begin{tabular}{@{}c@{}}
            请独立完成作业,不得抄袭。\\
            若参考了其它资料,请给出引用。\\
            鼓励讨论,但需独立书写解题过程。
        \end{tabular}}
    \end{center}
}

% Each hw consists of three parts:
% (1) this homework
\newcommand{\beginthishw}{\part{读书报告\faTasks}}
% (2) corrections (Optional)
\newcommand{\begincorrection}{\part{订正\faRefresh}}
% (3) any feedback (Optional)
\newcommand{\beginfb}{\part{反馈\faShareSquareO}}

% For math
\usepackage{amsmath}
\usepackage{amsfonts}
\usepackage{amssymb}
\usepackage{graphicx}
\usepackage{listings}
\usepackage{xcolor}
\usepackage{clrscode3e}
\usepackage{enumitem}
\usepackage{tikz}
\usepackage{tabularx}
\usepackage{multirow}
\newcolumntype{Y}{>{\centering\arraybackslash}X}
\newcolumntype{P}{>{\centering\arraybackslash}p}
%bigger integrate symbol
\usepackage{exscale}
\usepackage{relsize}
\usepackage{textcomp}

% colors
\newcommand{\red}[1]{\textcolor{red}{#1}}
\newcommand{\blue}[1]{\textcolor{blue}{#1}}
\newcommand{\teal}[1]{\textcolor{teal}{#1}}

% algorithms
\usepackage[]{algorithm}
\usepackage[noend]{algpseudocode} % noend
\makeatletter
\newenvironment{breakablealgorithm}
  {% \begin{breakablealgorithm}
      \begin{center}
          \refstepcounter{algorithm}% New algorithm
          \hrule height.8pt depth0pt \kern2pt% \@fs@pre for \@fs@ruled 画线
          \renewcommand{\caption}[2][\relax]{% Make a new \caption
              {\raggedright\textbf{\ALG@name~\thealgorithm} ##2\par}%
          \ifx\relax##1\relax % #1 is \relax
          \addcontentsline{loa}{algorithm}{\protect\numberline{\thealgorithm}##2}%
          \else % #1 is not \relax
          \addcontentsline{loa}{algorithm}{\protect\numberline{\thealgorithm}##1}%
          \fi
          \kern2pt\hrule\kern2pt
          }
          }{% \end{breakablealgorithm}
              \kern2pt\hrule\relax% \@fs@post for \@fs@ruled 画线
  \end{center}
  }
\makeatother
\renewcommand{\algorithmicrequire}{\textbf{Input:}} % Use Input in the format of Algorithm
\renewcommand{\algorithmicensure}{\textbf{Output:}} % Use Output in the format of Algorithm
% See [Adjust the indentation whithin the algorithmicx-package when a line is broken](https://tex.stackexchange.com/a/68540/23098)
\newcommand{\algparbox}[1]{\parbox[t]{\dimexpr\linewidth-\algorithmicindent}{#1\strut}}
\newcommand{\hStatex}[0]{\vspace{5pt}}
\makeatletter
\newlength{\trianglerightwidth}
\settowidth{\trianglerightwidth}{$\triangleright$~}
\algnewcommand{\LineComment}[1]{\Statex \hskip\ALG@thistlm triangleright#1}
\algnewcommand{\LineCommentCont}[1]{\Statex \hskip\ALG@thistlm%
  \parbox[t]{\dimexpr\linewidth-\ALG@thistlm}{\hangindent=\trianglerightwidth \hangafter=1 \strut$\triangleright$ #1\strut}}
\makeatother

% Define theorem-like environments
\usepackage[amsmath, thmmarks, framed]{ntheorem}
\usepackage{framed}

\theoremheaderfont{\kai\bfseries}
\theoremstyle{break}
% \theorembodyfont{\song}
\theorembodyfont{\kai}
\theoremseparator{\vspace{1mm}}
% \renewcommand*\FrameCommand{{\color{gray}\vrule width 3pt \hspace{10pt}}}
\renewcommand\FrameCommand{\color{gray}\vrule width 3pt \hspace{10pt}}
% \newtheorem*{problem}{\faCheckSquareO}
\newframedtheorem{problem}{\faCheckSquareO}

\theorempostwork{\hrule}
\newtheorem*{solution}{\faEdit}
\newtheorem*{revision}{\faEdit \ Revision}

\theoremstyle{plain}
\newtheorem*{cause}{\faCoffee \ Cause}
\newtheorem*{remark}{\faCommentingO \ Remark}

\theoremstyle{break}
\theoremsymbol{\ensuremath{\Box}}
\newtheorem*{proof}{\faEdit \ Proof}

\renewcommand\figurename{Figure}
\renewcommand\tablename{Table}

%enumeration
\setenumerate[1]{
    itemsep=0pt,
    partopsep=0pt,
    parsep=\parskip,
    topsep=0pt,
    leftmargin=20pt
}
\setitemize[1]{
    itemsep=0pt,
    partopsep=0pt,
    parsep=\parskip,
    topsep=0pt,
    leftmargin=20pt
}
\setdescription{
    itemsep=0pt,
    partopsep=0pt,
    parsep=\parskip,
    topsep=0pt,
    leftmargin=20pt
}
\lstset{
    language={[ISO]C++},
    numbers=left,
    numberstyle= \tiny,
    commentstyle=\color{red!50!green!50!blue!50},
    rulesepcolor=\color{red!20!green!20!blue!20},
    keywordstyle=\color{blue!90}\bfseries,
    showstringspaces=false,
    stringstyle=\ttfamily,
}


% For figures
% for fig with caption: #1: width/size; #2: fig file; #3: fig caption
\newcommand{\fig}[3]{
    \centerline{\includegraphics[scale=#1]{#2}}
    \centerline{#3}
}

% for fig without caption: #1: width/size; #2: fig file
\newcommand{\fignc}[2]{
    \centerline{\includegraphics[scale=#1]{#2}}
}

