%what is the homework? put it here.

%%%%%%%%%%%%%%%%%%%%%%%%%%%%%%%%%%%%%%%%%%%%%%%%%%%%%%%%%%%%%%%%
%                         My Template:                         %
%%%%%%%%%%%%%%%%%%%%%%%%%%%%%%%%%%%%%%%%%%%%%%%%%%%%%%%%%%%%%%%%

%Code(C++): \begin{lstlisting}
%Algorithm:
%\begin{breakablealgorithm}
%  \caption{?statement}
%  \begin{algorithmic}[?number]
%    \Require ?input
%    \Ensure ?output
%    \Procedure{Equal}{?parameters}
%      \State ?blabla
%    \EndProcedure
%  \end{algorithmic}
%\end{breakablealgorithm}

%Itemlisting: \begin{itemize} or \begin{enumerate}[label=(\alph*)]

%Math equation: \begin{align*}

%Table: \begin{tabular}{|c|c|c|}
%           blabla | blabla | blabla \\
%           ......
%Picture: \centerline{\includegraphics[scale=X]{FileName}

%%%%%%%%%%%%%%%%%%%%%%%%%%%%%%%%%%%%%%%%%%%%%%%%%%%%%%%%%%%%%%%%
%                         Title START!                         %
%%%%%%%%%%%%%%%%%%%%%%%%%%%%%%%%%%%%%%%%%%%%%%%%%%%%%%%%%%%%%%%%
\documentclass[11pt, a4paper, UTF8]{ctexart}

    \input{preamble}

    \title{经典悦读《红楼梦》}
    \me{毕秋宇}{计算机科学与技术系}{171860624}
    \date{\today}

    \begin{document}
    \maketitle
    % \noplagiarism

    %%%%%%%%%%%%%%%%%%%%%%%%%%%%%%%%%%%%%%%%%%%%%%%%%%%%%%%%%%%%%%%%
    %                       Homework START!                        %
    %%%%%%%%%%%%%%%%%%%%%%%%%%%%%%%%%%%%%%%%%%%%%%%%%%%%%%%%%%%%%%%%
    \beginthishw
    %%%%%%%%%%%%%%%%%%%%
    \begin{problem}[1-8回]
    \end{problem}

    %\begin{remark}
    %
    %\end{remark}
    \begin{solution}
        \large\indent 故事从木石前盟讲起,古云“天地生人,除大仁大恶,余者皆无大异;若大仁者则应运而生,大恶者则应劫而生,运生世治,劫生世危”,宝玉衔玉而生却落于“内囊尽上欲破侯府”,是运是劫一时难断。从“甄士隐梦幻识通灵”到“葫芦僧判断葫芦案”,贾雨村受恩人之女“英莲”被拐一案而不作为。侧面彰显了四大家族一荣俱荣、一损俱损的命运及环环相护、徇私枉法之现状,为最后“水满则溢,月盈则亏”两府一片白茫茫大地真干净留下伏笔。第三回,林黛玉抛父进京都,男女主角初相见,宝玉怒而摔玉,开启了两人相互纠缠的命运。
    \end{solution}

    % \begin{problem}[TJ: 9.23]
    % \end{problem}

    %\begin{remark}
    %
    %\end{remark}

    % \begin{solution}
    % \end{solution}

    % \begin{problem}[TJ: 10.11]
    % \end{problem}

    %\begin{remark}
    %
    %\end{remark}

    % \begin{solution}
    % \end{solution}

    % \begin{problem}[TJ: 11.5]
    % \end{problem}

    %\begin{remark}
    %
    %\end{remark}

    % \begin{solution}
    % \end{solution}

    % \begin{problem}[optional question 1: TJ: 11.17]
    % \end{problem}

    %\begin{remark}
    %
    %\end{remark}

    % \begin{solution}
    % \end{solution}

    % \begin{problem}[optional question 2]
    % \end{problem}

    %\begin{remark}
    %
    %\end{remark}

    % \begin{solution}
    % \end{solution}
    %%%%%%%%%%%%%%%%%%%%
    %\newpage
    %%%%%%%%%%%%%%%%%%%%


    %%%%%%%%%%%%%%%%%%%%%%%%%%%%%%%%%%%%%%%%%%%%%%%%%%%%%%%%%%%%%%%%
    %                      Correction START!                       %
    %%%%%%%%%%%%%%%%%%%%%%%%%%%%%%%%%%%%%%%%%%%%%%%%%%%%%%%%%%%%%%%%
    % \begincorrection
    %%%%%%%%%%%%%%%%%%%%
    %\begin{problem}[]

    %\end{problem}

    %\begin{cause}
    %
    %\end{cause}

    %\begin{revision}

    %\end{revision}
    %%%%%%%%%%%%%%%%%%%%
    %\newpage
    %%%%%%%%%%%%%%%%%%%%


    %%%%%%%%%%%%%%%%%%%%%%%%%%%%%%%%%%%%%%%%%%%%%%%%%%%%%%%%%%%%%%%%
    %                       Feedback START!                        %
    %%%%%%%%%%%%%%%%%%%%%%%%%%%%%%%%%%%%%%%%%%%%%%%%%%%%%%%%%%%%%%%%
    % \beginfb
    %\begin{itemize}
    %
    %\end{itemize}


    %%%%%%%%%%%%%%%%%%%%%%%%%%%%%%%%%%%%%%%%%%%%%%%%%%%%%%%%%%%%%%%%
    %                        Homework END!                         %
    %%%%%%%%%%%%%%%%%%%%%%%%%%%%%%%%%%%%%%%%%%%%%%%%%%%%%%%%%%%%%%%%
\end{document}

