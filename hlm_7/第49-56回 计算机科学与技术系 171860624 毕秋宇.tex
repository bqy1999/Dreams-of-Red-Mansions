%what is the homework? put it here.

%%%%%%%%%%%%%%%%%%%%%%%%%%%%%%%%%%%%%%%%%%%%%%%%%%%%%%%%%%%%%%%%
%                         My Template:                         %
%%%%%%%%%%%%%%%%%%%%%%%%%%%%%%%%%%%%%%%%%%%%%%%%%%%%%%%%%%%%%%%%

%Code(C++): \begin{lstlisting}
%Algorithm:
%\begin{breakablealgorithm}
%  \caption{?statement}
%  \begin{algorithmic}[?number]
%    \Require ?input
%    \Ensure ?output
%    \Procedure{Equal}{?parameters}
%      \State ?blabla
%    \EndProcedure
%  \end{algorithmic}
%\end{breakablealgorithm}

%Itemlisting: \begin{itemize} or \begin{enumerate}[label=(\alph*)]

%Math equation: \begin{align*}

%Table: \begin{tabular}{|c|c|c|}
%           blabla | blabla | blabla \\
%           ......
%Picture: \centerline{\includegraphics[scale=X]{FileName}

%%%%%%%%%%%%%%%%%%%%%%%%%%%%%%%%%%%%%%%%%%%%%%%%%%%%%%%%%%%%%%%%
%                         Title START!                         %
%%%%%%%%%%%%%%%%%%%%%%%%%%%%%%%%%%%%%%%%%%%%%%%%%%%%%%%%%%%%%%%%
\documentclass[11pt, a4paper, UTF8]{ctexart}

\input{preamble}

\title{经典悦读《红楼梦》}
\me{毕秋宇}{计算机科学与技术系}{171860624}
\date{\today}

\begin{document}
\maketitle
% \noplagiarism

%%%%%%%%%%%%%%%%%%%%%%%%%%%%%%%%%%%%%%%%%%%%%%%%%%%%%%%%%%%%%%%%
%                       Homework START!                        %
%%%%%%%%%%%%%%%%%%%%%%%%%%%%%%%%%%%%%%%%%%%%%%%%%%%%%%%%%%%%%%%%
\beginthishw
%%%%%%%%%%%%%%%%%%%%
\begin{problem}[第49-56回]
\end{problem}

%\begin{remark}
%
%\end{remark}
\begin{solution}
    \large 妙玉,私以为《红楼梦》中真真正正一位奇女子,在第四十一回一闪而过之后,时隔八回方才再次登场。第四十九回写宝玉雪中赏梅,“琉璃世界白雪红梅”好一派洁净美好的模样,常听人言妙玉如梅,本回之中可见一斑。妙玉生在佛门,内心孤傲纯洁,可谓是琉璃世界,但她也是一位妙龄少女,怀揣着自己的情愫,所以又与白梅不同是誉为红梅。气质纯洁,内心热烈,妙玉正是这对矛盾的巧妙结合体,从本回中可以看出妙玉对宝玉暗怀的情愫,但她终身守于尼姑庵,注定一生孤苦伶仃,与情无缘。再想到这么一枚无瑕白玉,最后竟落得了一个“无瑕白玉遭泥陷,王孙公子叹无缘”,不由得悲从中来。再与五十三回荣国府众人荒淫无度,入不敷出,一次省亲花费无数做对比,不由想起“天道不公,叫无才无德者为富贵”。
\end{solution}

% \begin{problem}[TJ: 9.23]
% \end{problem}

%\begin{remark}
%
%\end{remark}

% \begin{solution}
% \end{solution}

% \begin{problem}[TJ: 10.11]
% \end{problem}

%\begin{remark}
%
%\end{remark}

% \begin{solution}
% \end{solution}

% \begin{problem}[TJ: 11.5]
% \end{problem}

%\begin{remark}
%
%\end{remark}

% \begin{solution}
% \end{solution}

% \begin{problem}[optional question 1: TJ: 11.17]
% \end{problem}

%\begin{remark}
%
%\end{remark}

% \begin{solution}
% \end{solution}

% \begin{problem}[optional question 2]
% \end{problem}

%\begin{remark}
%
%\end{remark}

% \begin{solution}
% \end{solution}
%%%%%%%%%%%%%%%%%%%%
%\newpage
%%%%%%%%%%%%%%%%%%%%


%%%%%%%%%%%%%%%%%%%%%%%%%%%%%%%%%%%%%%%%%%%%%%%%%%%%%%%%%%%%%%%%
%                      Correction START!                       %
%%%%%%%%%%%%%%%%%%%%%%%%%%%%%%%%%%%%%%%%%%%%%%%%%%%%%%%%%%%%%%%%
% \begincorrection
%%%%%%%%%%%%%%%%%%%%
%\begin{problem}[]

%\end{problem}

%\begin{cause}
%
%\end{cause}

%\begin{revision}

%\end{revision}
%%%%%%%%%%%%%%%%%%%%
%\newpage
%%%%%%%%%%%%%%%%%%%%


%%%%%%%%%%%%%%%%%%%%%%%%%%%%%%%%%%%%%%%%%%%%%%%%%%%%%%%%%%%%%%%%
%                       Feedback START!                        %
%%%%%%%%%%%%%%%%%%%%%%%%%%%%%%%%%%%%%%%%%%%%%%%%%%%%%%%%%%%%%%%%
% \beginfb
%\begin{itemize}
%
%\end{itemize}


%%%%%%%%%%%%%%%%%%%%%%%%%%%%%%%%%%%%%%%%%%%%%%%%%%%%%%%%%%%%%%%%
%                        Homework END!                         %
%%%%%%%%%%%%%%%%%%%%%%%%%%%%%%%%%%%%%%%%%%%%%%%%%%%%%%%%%%%%%%%%
\end{document}

