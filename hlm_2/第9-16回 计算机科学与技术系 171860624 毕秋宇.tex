%what is the homework? put it here.

%%%%%%%%%%%%%%%%%%%%%%%%%%%%%%%%%%%%%%%%%%%%%%%%%%%%%%%%%%%%%%%%
%                         My Template:                         %
%%%%%%%%%%%%%%%%%%%%%%%%%%%%%%%%%%%%%%%%%%%%%%%%%%%%%%%%%%%%%%%%

%Code(C++): \begin{lstlisting}
%Algorithm:
%\begin{breakablealgorithm}
%  \caption{?statement}
%  \begin{algorithmic}[?number]
%    \Require ?input
%    \Ensure ?output
%    \Procedure{Equal}{?parameters}
%      \State ?blabla
%    \EndProcedure
%  \end{algorithmic}
%\end{breakablealgorithm}

%Itemlisting: \begin{itemize} or \begin{enumerate}[label=(\alph*)]

%Math equation: \begin{align*}

%Table: \begin{tabular}{|c|c|c|}
%           blabla | blabla | blabla \\
%           ......
%Picture: \centerline{\includegraphics[scale=X]{FileName}

%%%%%%%%%%%%%%%%%%%%%%%%%%%%%%%%%%%%%%%%%%%%%%%%%%%%%%%%%%%%%%%%
%                         Title START!                         %
%%%%%%%%%%%%%%%%%%%%%%%%%%%%%%%%%%%%%%%%%%%%%%%%%%%%%%%%%%%%%%%%
\documentclass[11pt, a4paper, UTF8]{ctexart}

    \input{preamble}

    \title{经典悦读《红楼梦》}
    \me{毕秋宇}{计算机科学与技术系}{171860624}
    \date{\today}

    \begin{document}
    \maketitle
    % \noplagiarism

    %%%%%%%%%%%%%%%%%%%%%%%%%%%%%%%%%%%%%%%%%%%%%%%%%%%%%%%%%%%%%%%%
    %                       Homework START!                        %
    %%%%%%%%%%%%%%%%%%%%%%%%%%%%%%%%%%%%%%%%%%%%%%%%%%%%%%%%%%%%%%%%
    \beginthishw
    %%%%%%%%%%%%%%%%%%%%
    \begin{problem}[9-16回]
    \end{problem}

    %\begin{remark}
    %
    %\end{remark}
    \begin{solution}
        \large 九回“顽童闹书房”初显宝玉不适走“学优则仕”之道,不同于前篇他人之言,本回直写“劣子”“顽童”之态,暗含所谓名门望族外强中干后继无人之患。宝玉求学实为欲与秦鲸卿相会,同窗因嫌隙闯祸段,小子矛盾尽能惹得整个家族争锋相对,一时丑态跃然纸上。十三回秦可卿死封龙王尉,十二钗初显退场,“漫言不肖皆荣出,造衅开端实在宁”,多情才女秦可卿隐恶成疾一命呜呼,借人世间风月情多,说大族复杂纠葛。后又“凤姐弄权”更显其肆意妄为、草芥人命,为坐享银两不顾屈死性命。秦鲸卿与智能儿得趣仅一回之后也遂姊而去,令人唏嘘,真是心狠手辣得权势,未语先红自误了。鲸卿夭逝前悔悟自恃才高,劝宝玉立志功名,无奈大势已定,也无憾动之力了。
    \end{solution}

    % \begin{problem}[TJ: 9.23]
    % \end{problem}

    %\begin{remark}
    %
    %\end{remark}

    % \begin{solution}
    % \end{solution}

    % \begin{problem}[TJ: 10.11]
    % \end{problem}

    %\begin{remark}
    %
    %\end{remark}

    % \begin{solution}
    % \end{solution}

    % \begin{problem}[TJ: 11.5]
    % \end{problem}

    %\begin{remark}
    %
    %\end{remark}

    % \begin{solution}
    % \end{solution}

    % \begin{problem}[optional question 1: TJ: 11.17]
    % \end{problem}

    %\begin{remark}
    %
    %\end{remark}

    % \begin{solution}
    % \end{solution}

    % \begin{problem}[optional question 2]
    % \end{problem}

    %\begin{remark}
    %
    %\end{remark}

    % \begin{solution}
    % \end{solution}
    %%%%%%%%%%%%%%%%%%%%
    %\newpage
    %%%%%%%%%%%%%%%%%%%%


    %%%%%%%%%%%%%%%%%%%%%%%%%%%%%%%%%%%%%%%%%%%%%%%%%%%%%%%%%%%%%%%%
    %                      Correction START!                       %
    %%%%%%%%%%%%%%%%%%%%%%%%%%%%%%%%%%%%%%%%%%%%%%%%%%%%%%%%%%%%%%%%
    % \begincorrection
    %%%%%%%%%%%%%%%%%%%%
    %\begin{problem}[]

    %\end{problem}

    %\begin{cause}
    %
    %\end{cause}

    %\begin{revision}

    %\end{revision}
    %%%%%%%%%%%%%%%%%%%%
    %\newpage
    %%%%%%%%%%%%%%%%%%%%


    %%%%%%%%%%%%%%%%%%%%%%%%%%%%%%%%%%%%%%%%%%%%%%%%%%%%%%%%%%%%%%%%
    %                       Feedback START!                        %
    %%%%%%%%%%%%%%%%%%%%%%%%%%%%%%%%%%%%%%%%%%%%%%%%%%%%%%%%%%%%%%%%
    % \beginfb
    %\begin{itemize}
    %
    %\end{itemize}


    %%%%%%%%%%%%%%%%%%%%%%%%%%%%%%%%%%%%%%%%%%%%%%%%%%%%%%%%%%%%%%%%
    %                        Homework END!                         %
    %%%%%%%%%%%%%%%%%%%%%%%%%%%%%%%%%%%%%%%%%%%%%%%%%%%%%%%%%%%%%%%%
\end{document}

