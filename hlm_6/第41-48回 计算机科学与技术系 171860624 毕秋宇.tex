%what is the homework? put it here.

%%%%%%%%%%%%%%%%%%%%%%%%%%%%%%%%%%%%%%%%%%%%%%%%%%%%%%%%%%%%%%%%
%                         My Template:                         %
%%%%%%%%%%%%%%%%%%%%%%%%%%%%%%%%%%%%%%%%%%%%%%%%%%%%%%%%%%%%%%%%

%Code(C++): \begin{lstlisting}
%Algorithm:
%\begin{breakablealgorithm}
%  \caption{?statement}
%  \begin{algorithmic}[?number]
%    \Require ?input
%    \Ensure ?output
%    \Procedure{Equal}{?parameters}
%      \State ?blabla
%    \EndProcedure
%  \end{algorithmic}
%\end{breakablealgorithm}

%Itemlisting: \begin{itemize} or \begin{enumerate}[label=(\alph*)]

%Math equation: \begin{align*}

%Table: \begin{tabular}{|c|c|c|}
%           blabla | blabla | blabla \\
%           ......
%Picture: \centerline{\includegraphics[scale=X]{FileName}

%%%%%%%%%%%%%%%%%%%%%%%%%%%%%%%%%%%%%%%%%%%%%%%%%%%%%%%%%%%%%%%%
%                         Title START!                         %
%%%%%%%%%%%%%%%%%%%%%%%%%%%%%%%%%%%%%%%%%%%%%%%%%%%%%%%%%%%%%%%%
\documentclass[11pt, a4paper, UTF8]{ctexart}

\input{preamble}

\title{经典悦读《红楼梦》}
\me{毕秋宇}{计算机科学与技术系}{171860624}
\date{\today}

\begin{document}
\maketitle
% \noplagiarism

%%%%%%%%%%%%%%%%%%%%%%%%%%%%%%%%%%%%%%%%%%%%%%%%%%%%%%%%%%%%%%%%
%                       Homework START!                        %
%%%%%%%%%%%%%%%%%%%%%%%%%%%%%%%%%%%%%%%%%%%%%%%%%%%%%%%%%%%%%%%%
\beginthishw
%%%%%%%%%%%%%%%%%%%%
\begin{problem}[第41-48回]
\end{problem}

%\begin{remark}
%
%\end{remark}
\begin{solution}
    \large 第43回中,宝玉假借他人名义给金钏儿烧纸,可见其重情重义和对悲情女子表现出的同情怜悯,即使冒着“不孝之名”也要为薄情女子抱不平,这为后来家境衰微宝玉出家埋下伏笔。下一回中,凤姐听闻贾琏偷情、还欲扶正平儿怒从心生,宝钗袭人对平儿连番相劝,宝玉见平儿被打不由想到其命较黛玉更薄,为之落泪,由此宝玉性格更加凸显。最终由贾母从中协调事情才得以解决,经此一事,凤姐之要强泼辣,贾琏之淫乱荒诞,平儿之命如纸薄跃然纸上,可谓是“主子吵架、奴才受灾”。后与贾琏偷情的鲍二媳妇上吊,贾家竟要由王子腾的强势帮助解决,贾家之衰落也可见一斑。
\end{solution}

% \begin{problem}[TJ: 9.23]
% \end{problem}

%\begin{remark}
%
%\end{remark}

% \begin{solution}
% \end{solution}

% \begin{problem}[TJ: 10.11]
% \end{problem}

%\begin{remark}
%
%\end{remark}

% \begin{solution}
% \end{solution}

% \begin{problem}[TJ: 11.5]
% \end{problem}

%\begin{remark}
%
%\end{remark}

% \begin{solution}
% \end{solution}

% \begin{problem}[optional question 1: TJ: 11.17]
% \end{problem}

%\begin{remark}
%
%\end{remark}

% \begin{solution}
% \end{solution}

% \begin{problem}[optional question 2]
% \end{problem}

%\begin{remark}
%
%\end{remark}

% \begin{solution}
% \end{solution}
%%%%%%%%%%%%%%%%%%%%
%\newpage
%%%%%%%%%%%%%%%%%%%%


%%%%%%%%%%%%%%%%%%%%%%%%%%%%%%%%%%%%%%%%%%%%%%%%%%%%%%%%%%%%%%%%
%                      Correction START!                       %
%%%%%%%%%%%%%%%%%%%%%%%%%%%%%%%%%%%%%%%%%%%%%%%%%%%%%%%%%%%%%%%%
% \begincorrection
%%%%%%%%%%%%%%%%%%%%
%\begin{problem}[]

%\end{problem}

%\begin{cause}
%
%\end{cause}

%\begin{revision}

%\end{revision}
%%%%%%%%%%%%%%%%%%%%
%\newpage
%%%%%%%%%%%%%%%%%%%%


%%%%%%%%%%%%%%%%%%%%%%%%%%%%%%%%%%%%%%%%%%%%%%%%%%%%%%%%%%%%%%%%
%                       Feedback START!                        %
%%%%%%%%%%%%%%%%%%%%%%%%%%%%%%%%%%%%%%%%%%%%%%%%%%%%%%%%%%%%%%%%
% \beginfb
%\begin{itemize}
%
%\end{itemize}


%%%%%%%%%%%%%%%%%%%%%%%%%%%%%%%%%%%%%%%%%%%%%%%%%%%%%%%%%%%%%%%%
%                        Homework END!                         %
%%%%%%%%%%%%%%%%%%%%%%%%%%%%%%%%%%%%%%%%%%%%%%%%%%%%%%%%%%%%%%%%
\end{document}

