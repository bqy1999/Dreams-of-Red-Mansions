%what is the homework? put it here.

%%%%%%%%%%%%%%%%%%%%%%%%%%%%%%%%%%%%%%%%%%%%%%%%%%%%%%%%%%%%%%%%
%                         My Template:                         %
%%%%%%%%%%%%%%%%%%%%%%%%%%%%%%%%%%%%%%%%%%%%%%%%%%%%%%%%%%%%%%%%

%Code(C++): \begin{lstlisting}
%Algorithm:
%\begin{breakablealgorithm}
%  \caption{?statement}
%  \begin{algorithmic}[?number]
%    \Require ?input
%    \Ensure ?output
%    \Procedure{Equal}{?parameters}
%      \State ?blabla
%    \EndProcedure
%  \end{algorithmic}
%\end{breakablealgorithm}

%Itemlisting: \begin{itemize} or \begin{enumerate}[label=(\alph*)]

%Math equation: \begin{align*}

%Table: \begin{tabular}{|c|c|c|}
%           blabla | blabla | blabla \\
%           ......
%Picture: \centerline{\includegraphics[scale=X]{FileName}

%%%%%%%%%%%%%%%%%%%%%%%%%%%%%%%%%%%%%%%%%%%%%%%%%%%%%%%%%%%%%%%%
%                         Title START!                         %
%%%%%%%%%%%%%%%%%%%%%%%%%%%%%%%%%%%%%%%%%%%%%%%%%%%%%%%%%%%%%%%%
\documentclass[11pt, a4paper, UTF8]{ctexart}

\input{preamble}

\title{经典悦读《红楼梦》}
\me{毕秋宇}{计算机科学与技术系}{171860624}
\date{\today}

\begin{document}
\maketitle
% \noplagiarism

%%%%%%%%%%%%%%%%%%%%%%%%%%%%%%%%%%%%%%%%%%%%%%%%%%%%%%%%%%%%%%%%
%                       Homework START!                        %
%%%%%%%%%%%%%%%%%%%%%%%%%%%%%%%%%%%%%%%%%%%%%%%%%%%%%%%%%%%%%%%%
\beginthishw
%%%%%%%%%%%%%%%%%%%%
\begin{problem}[第73-80回]
\end{problem}

%\begin{remark}
%
%\end{remark}
\begin{solution}
    \

    \large 第七十四回迎来了本书另一个大的高潮“谗奸惑抄捡大观园 避嫌隙杜绝宁国府”。王善宝的妻子向王夫人进谗言,在背后诬陷晴雯。实际上是主子邢夫人为了削弱王夫人的实力,想的一记“妙招”。邢夫人所说看似名正言顺,王夫人不敢不照办,为了查处绣春囊,对大观园的丫鬟进行了搜查。\\
    家族之内斗到了如此地步,这贾家还如何能够兴旺发达呢。平日里跟丫头女子混在一起的贾宝玉,在王宝善妻子一行人来抄检的时候,竟是如此的窝囊,真叫人罔欣赏了之前的重情重义。倒是探春,性情刚烈,有凌人之气,一巴掌,教读者大呼“过瘾”。\\
    无奈,司棋的情书被搜出,当司棋拉着宝玉的手,痛哭着求救的时候,无能的宝玉也只能报以痛哭。她被逐出贾府,是谁的过失,我不知道,但是细读这段文字,不由的悲从中来。大观园被抄,留得一地鸡毛,司棋被逐后死去,惜春与宁国府绝交,这是一场双输的赌局,两方,都没有捞到好处。\\
    接下来几回一回惨过一回,叫人难以接受。先是晴雯早夭,含冤逝世,再是薛蟠误娶夏金桂,终迎春错嫁孙绍祖,被虐致死。教读者既悲又恨,有时甚至想冲进书里叫那孙绍祖好看。直到最后,才看清了红楼烟雨下封建社会吃人的本质,歹毒的权谋家能够得势,纯洁神圣的佳人才子却只有悲剧下场,可悲可叹。
\end{solution}

% \begin{problem}[TJ: 9.23]
% \end{problem}

%\begin{remark}
%
%\end{remark}

% \begin{solution}
% \end{solution}

% \begin{problem}[TJ: 10.11]
% \end{problem}

%\begin{remark}
%
%\end{remark}

% \begin{solution}
% \end{solution}

% \begin{problem}[TJ: 11.5]
% \end{problem}

%\begin{remark}
%
%\end{remark}

% \begin{solution}
% \end{solution}

% \begin{problem}[optional question 1: TJ: 11.17]
% \end{problem}

%\begin{remark}
%
%\end{remark}

% \begin{solution}
% \end{solution}

% \begin{problem}[optional question 2]
% \end{problem}

%\begin{remark}
%
%\end{remark}

% \begin{solution}
% \end{solution}
%%%%%%%%%%%%%%%%%%%%
%\newpage
%%%%%%%%%%%%%%%%%%%%


%%%%%%%%%%%%%%%%%%%%%%%%%%%%%%%%%%%%%%%%%%%%%%%%%%%%%%%%%%%%%%%%
%                      Correction START!                       %
%%%%%%%%%%%%%%%%%%%%%%%%%%%%%%%%%%%%%%%%%%%%%%%%%%%%%%%%%%%%%%%%
% \begincorrection
%%%%%%%%%%%%%%%%%%%%
%\begin{problem}[]

%\end{problem}

%\begin{cause}
%
%\end{cause}

%\begin{revision}

%\end{revision}
%%%%%%%%%%%%%%%%%%%%
%\newpage
%%%%%%%%%%%%%%%%%%%%


%%%%%%%%%%%%%%%%%%%%%%%%%%%%%%%%%%%%%%%%%%%%%%%%%%%%%%%%%%%%%%%%
%                       Feedback START!                        %
%%%%%%%%%%%%%%%%%%%%%%%%%%%%%%%%%%%%%%%%%%%%%%%%%%%%%%%%%%%%%%%%
% \beginfb
%\begin{itemize}
%
%\end{itemize}


%%%%%%%%%%%%%%%%%%%%%%%%%%%%%%%%%%%%%%%%%%%%%%%%%%%%%%%%%%%%%%%%
%                        Homework END!                         %
%%%%%%%%%%%%%%%%%%%%%%%%%%%%%%%%%%%%%%%%%%%%%%%%%%%%%%%%%%%%%%%%
\end{document}

