%what is the homework? put it here.

%%%%%%%%%%%%%%%%%%%%%%%%%%%%%%%%%%%%%%%%%%%%%%%%%%%%%%%%%%%%%%%%
%                         My Template:                         %
%%%%%%%%%%%%%%%%%%%%%%%%%%%%%%%%%%%%%%%%%%%%%%%%%%%%%%%%%%%%%%%%

%Code(C++): \begin{lstlisting}
%Algorithm:
%\begin{breakablealgorithm}
%  \caption{?statement}
%  \begin{algorithmic}[?number]
%    \Require ?input
%    \Ensure ?output
%    \Procedure{Equal}{?parameters}
%      \State ?blabla
%    \EndProcedure
%  \end{algorithmic}
%\end{breakablealgorithm}

%Itemlisting: \begin{itemize} or \begin{enumerate}[label=(\alph*)]

%Math equation: \begin{align*}

%Table: \begin{tabular}{|c|c|c|}
%           blabla | blabla | blabla \\
%           ......
%Picture: \centerline{\includegraphics[scale=X]{FileName}

%%%%%%%%%%%%%%%%%%%%%%%%%%%%%%%%%%%%%%%%%%%%%%%%%%%%%%%%%%%%%%%%
%                         Title START!                         %
%%%%%%%%%%%%%%%%%%%%%%%%%%%%%%%%%%%%%%%%%%%%%%%%%%%%%%%%%%%%%%%%
\documentclass[11pt, a4paper, UTF8]{ctexart}

\input{preamble}

\title{经典悦读《红楼梦》}
\me{毕秋宇}{计算机科学与技术系}{171860624}
\date{\today}

\begin{document}
\maketitle
% \noplagiarism

%%%%%%%%%%%%%%%%%%%%%%%%%%%%%%%%%%%%%%%%%%%%%%%%%%%%%%%%%%%%%%%%
%                       Homework START!                        %
%%%%%%%%%%%%%%%%%%%%%%%%%%%%%%%%%%%%%%%%%%%%%%%%%%%%%%%%%%%%%%%%
\beginthishw
%%%%%%%%%%%%%%%%%%%%
\begin{problem}[25-32回]
\end{problem}

%\begin{remark}
%
%\end{remark}
\begin{solution}
    \large 个人认为这八回中最精彩的莫过于“黛玉葬花”,黛玉之敏感,宝黛之性情相投显露无疑。二十六回中,黛玉寻宝玉未被准入而心生寄人篱下之苦,又闻得宝玉宝钗交谈之声,更加悲从中来,呜咽不止,看落花缤纷后作葬花词。翌日宝玉听闻葬花词,发生了共鸣,一想到黛玉花容有朝一日不复存在尽瘫倒在地。葬花词中黛玉借花自比,充分展现出了她作为寄人篱下的女子的孤苦伶仃,而宝玉恸惊天地也反映了与黛玉的情义相合。宝黛借此一扫前嫌,两人经此一事关系更甚与之前,在薛蟠生日之际两冤家相聚头,宝玉说出若黛玉去世将出家为僧,也算是为最后的结局埋下伏笔了。
\end{solution}

% \begin{problem}[TJ: 9.23]
% \end{problem}

%\begin{remark}
%
%\end{remark}

% \begin{solution}
% \end{solution}

% \begin{problem}[TJ: 10.11]
% \end{problem}

%\begin{remark}
%
%\end{remark}

% \begin{solution}
% \end{solution}

% \begin{problem}[TJ: 11.5]
% \end{problem}

%\begin{remark}
%
%\end{remark}

% \begin{solution}
% \end{solution}

% \begin{problem}[optional question 1: TJ: 11.17]
% \end{problem}

%\begin{remark}
%
%\end{remark}

% \begin{solution}
% \end{solution}

% \begin{problem}[optional question 2]
% \end{problem}

%\begin{remark}
%
%\end{remark}

% \begin{solution}
% \end{solution}
%%%%%%%%%%%%%%%%%%%%
%\newpage
%%%%%%%%%%%%%%%%%%%%


%%%%%%%%%%%%%%%%%%%%%%%%%%%%%%%%%%%%%%%%%%%%%%%%%%%%%%%%%%%%%%%%
%                      Correction START!                       %
%%%%%%%%%%%%%%%%%%%%%%%%%%%%%%%%%%%%%%%%%%%%%%%%%%%%%%%%%%%%%%%%
% \begincorrection
%%%%%%%%%%%%%%%%%%%%
%\begin{problem}[]

%\end{problem}

%\begin{cause}
%
%\end{cause}

%\begin{revision}

%\end{revision}
%%%%%%%%%%%%%%%%%%%%
%\newpage
%%%%%%%%%%%%%%%%%%%%


%%%%%%%%%%%%%%%%%%%%%%%%%%%%%%%%%%%%%%%%%%%%%%%%%%%%%%%%%%%%%%%%
%                       Feedback START!                        %
%%%%%%%%%%%%%%%%%%%%%%%%%%%%%%%%%%%%%%%%%%%%%%%%%%%%%%%%%%%%%%%%
% \beginfb
%\begin{itemize}
%
%\end{itemize}


%%%%%%%%%%%%%%%%%%%%%%%%%%%%%%%%%%%%%%%%%%%%%%%%%%%%%%%%%%%%%%%%
%                        Homework END!                         %
%%%%%%%%%%%%%%%%%%%%%%%%%%%%%%%%%%%%%%%%%%%%%%%%%%%%%%%%%%%%%%%%
\end{document}

